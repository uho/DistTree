\documentclass{article}

\usepackage[latin1]{inputenc}
\usepackage{times}
\parindent0pt
\parskip3pt

\newcommand{\disttree}{{\tt disttree}}

\title{\disttree: A Small Language for Creating Distribution Trees}
\author{Ulrich Hoffmann}
\date{May 17, 2002}

\begin{document}

\maketitle

\begin{abstract}
This document describes the syntax and semantics of the small
specification language \disttree\ which can be used to conviniently
specify the mapping from build trees to distribution trees for
software deployment. The reader will learn about the language elements
and, a processor for \disttree, and a small example.
\end{abstract}

\section{Introduction}
When large software systems are created, the task to finally deploy
the software on a target system is of significant work. Typically a
build process generates files in a directory structure (the {\em build
  tree}), which is structured according to software development needs.
The target system often requires a rather different directory
structure (the {\em distribution tree}) to successfully run the
software. It is necessary to have a mechanism to map from the build
tree to the distribution tree. The small language \disttree\ is such a
mechanism, that allows to specify how to construct a complete
distribution tree from a build tree. The use of a macro substitution
mechanism in the specification text allows for easy adaptation to
changes in both trees. Based on the specification a \disttree\ 
processor actually copies files fromt the build tree to their
appriopriate places in the distribution tree. The processor allows for
conventient generation of log/manifest files and is easily
integrated in a fully automated build process.

\section{\disttree\ Language Elements}
\disttree\ is designed to be as simple as possible in order to be
easily learned. It is as sophisticated as necessary to express the
various needs for convenient distribution tree generation.
\disttree\ has a line oriented syntax. Each language element has to
be written on a line of its own. The line lenght is not limited.

\subsection{Comments}
Comments in \disttree\ are important to explain the rationale and the
structure of the specification. A \disttree\ comment is a lines which
first non whitespace character is a number sign {\tt \#}. All
remaining characters until the end of line are ignored.

\subsection{Variable Assignments}
A \disttree\ variable identifier is a lower or uppercase letter, oder
{\tt \_} character possibly followed by additional lower-, uppercase
letters or {\tt \_}s or digits. Thus {\tt temp3}, {\tt ROOT}, {\tt
  CONFIG\_DATA} are all variable identifiers, whereas {\tt SRC-DIR} is
not. Case is significant.

\disttree\ assignments have the form variable{\tt =}$<$rhs$>$


rhs is a string, which possibly contains variable references of the
form {\tt \$}variable. The value of variable is expanded in order
to create a variable free string. This string is assigned to be the
value of the variable on the left hand side of the assignement.


\subsection{Target Directory Specifications}
A target directory {\tt [}path{\tt ]}

\subsection{Source File Specifications}
path

files and directories

\subsection{Message Output}
{\tt \%error }message

{\tt \%print }

{\tt \%print }message


\subsection{Conditional Processing}
{\tt \%ifdef }variable

{\tt \%ifndef }variable

{\tt \%else}

{\tt \%endif}


\subsection{Looped Processing}

{\tt \%for} variable1, variable2, ... {\tt in} sequence

{\tt \%endfor}



\subsection{External Command Execution}

{\tt @} command

\section{A Processor for \disttree}

\begin{verbatim}
Distribution Tree Creator
Create a distribution tree according to disttreespec

Usage: python disttree.py [options] disttreespec
       options:
          -v             verbose
          -q             on copy error, ask retry question
          -n             don't actually perform copy
          -w             just warn if a file does not exist
          -f             force (overwrite existing r/o files)
          -a             reset read-only attributes
          -c             don't create empty directories
          -m             generate md5sums in logfile
          -l logfile     write log file
          -D name=value  preset variable

Example: python disttree.py -v -D DST=X:\ distribution.spec
\end{verbatim}

\section{A sample distribution tree specification}

\begin{verbatim}
# A sample disttree script

# Notify user that generation has started
%print start

# SRC and DST are typically set when invoking
# the disttree processor
%ifndef DST
DST=
%endif

%ifndef SRC
%error SRC must be set using -D when invoking disttree
%endif

# Setup a target directory ROOT
ROOT=$DST\ROOT

# Define Source locations
LOCVOB=$SRC\localization
COMP1VOB=$SRC\comp1
C1BIN=$COMP1\bin
COMP2VOB=$SRC\comp2
C2BIN=$COMP2\bin
COMP3VOB=$SRC\comp3
C3BIN=$COMP3\bin

# Setup localization information
GERMAN=German,de
FRENCH=French,fr
ENGLISH=English,en

LANGUAGES=$GERMAN,$FRENCH,$ENGLISH

# iterate localization

%for LANG,L in $LANGUAGES
[$ROOT\locale\$LANG]
$LOCVOB\localizations\localization_$L.jar
%endfor

# Some executables go to the bin dir
[$ROOT\bin]
$C1BIN\component1.exe
# or similar
$COMP2VOB\bin\component2.exe
$COMP3VOB\bin\component3.exe

[$ROOT\lib]
# Mentining a directory recursively enumerates files
$COMP1VOB\lib

# Notify user that generation has ended
%print end

\end{verbatim}


\end{document}
